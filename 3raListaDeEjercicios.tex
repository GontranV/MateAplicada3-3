\documentclass[oneside]{article}
\usepackage{epsfig,graphicx} % Required for inserting images
\usepackage[linesnumbered,ruled,vlined]{algorithm2e}
\usepackage{amsmath}
\usepackage{amsthm}
\usepackage{amssymb}
\usepackage{subcaption}
\usepackage{xcolor}
%\usepackage[spanish,mexico]{babel}
\usepackage[bookmarksopen]{hyperref}
\usepackage[utf8]{inputenc}
\usepackage{array}
\usepackage{listings} %Soporte para código
\usepackage[left=2cm,right=2cm,top=1.8cm,bottom=2.3cm]{geometry}
\usepackage{titlesec}
\usepackage{fancyhdr} 
\usepackage{enumitem}
\usepackage{multicol}           % Permits header customization. See header section below.
\usepackage{wrapfig}
\usepackage{nicematrix} 
\usepackage[T1]{fontenc}
\usepackage{lmodern}
% \usepackage{graphicx}
\usepackage{color}
\usepackage{amsfonts}
\usepackage{epstopdf}
% \usepackage[table]{xcolor}
% \usepackage{matlab}
\usepackage{tikz}
% \documentclass[tikz,border=3mm]{standalone}

% \usepackage[paperheight=795pt,paperwidth=614pt,top=72pt,bottom=72pt,right=72pt,left=72pt,heightrounded]{geometry}

\definecolor{codegreen}{rgb}{0,0.6,0}
\definecolor{codegray}{rgb}{0.5,0.5,0.5}
\definecolor{codepurple}{rgb}{0.58,0,0.82}
\definecolor{backcolour}{rgb}{0.95,0.95,0.92}

\lstdefinestyle{mystyle}{
    backgroundcolor=\color{backcolour},   
    commentstyle=\color{codegreen},
    keywordstyle=\color{magenta},
    numberstyle=\tiny\color{codegray},
    stringstyle=\color{codepurple},
    basicstyle=\ttfamily\footnotesize,
    breakatwhitespace=false,         
    breaklines=true,                 
    captionpos=b,                    
    keepspaces=true,                 
    numbers=left,                    
    numbersep=5pt,                  
    showspaces=false,                
    showstringspaces=false,
    showtabs=false,                  
    tabsize=2,
    language=Java
}
\sloppy
\epstopdfsetup{outdir=./}
\graphicspath{ {./untitled6_media/} }

% Crear matrices más cool
% \usepackage[letterpaper,headheight=20pt]{geometry}
\addtolength{\textheight}{1.5cm}

\renewcommand{\theenumi}{\arabic{enumi}}
% \renewcommand{\labelenumi}{\theenumi}
\renewcommand{\headrulewidth}{0pt}
\newcommand{\raiz}[1]{\sqrt{#1}}
\newcommand{\norm}[1]{\big|\big|#1\big|\big|}
\newcommand{\diff}[2]{\frac{d#1}{d#2}}
\newcommand{\pdiff}[2]{\frac{\partial#1}{\partial#2}}
\newcommand{\pdiffdos}[2]{\frac{\partial^2 #1}{\partial #2^2}}
% ---definición de los paquetes--
\fancypagestyle{plain}{
    \lhead{}
    \fancyhead[R]{\thepage}
    \fancyhead[L]{}
    \renewcommand{\headrulewidth}{0pt}
    \fancyfoot{}
}
\pagestyle{fancyplain}

\newcommand{\diam}{\operatorname{diam}}
\fancyhead[R]{\thepage}
\fancyhead[L]{}
% Definir el tamaño del título del capítulo
\titleformat{\chapter}[display]
  {\normalfont\huge\bfseries} % Estilo del título
  {\chaptertitlename\ \thechapter}{1pt}{\Large} % Tamaño del título
  \titlespacing*{\chapter}{0pt}{-20pt}{20pt} % Ajustar el espaciado

% \title{2da lista de problemas}
% \author{Ramírez Mendoza Joaquín Rodrigo\\
% Villalobos Juárez Gontrán Eliut\\
% Treviño Puebla Héctor Jerome} 
% \date{\today}

\begin{document}

% ---Inicio de la portada
% \begin{titlepage}

%   \begin{minipage}{3cm}
%     \begin{center}
%       \includegraphics[height = 0.14\textheight]{recursos/Logo_UNAM.png}\par
%     \end{center}
%   \end{minipage}\hfill
%   \begin{minipage}{10cm}

%   \end{minipage}\hfill
%   \begin{minipage}{3cm}
%     \begin{center}
%       \includegraphics[height = 0.14\textheight]{recursos/Logo_FC.png}\par
%     \end{center}
%   \end{minipage}
%   \centering
%   \vspace{1cm}

%   {\bfseries\LARGE Universidad Nacional Autónoma de México \par}

%   \vspace{1cm}
%   {\scshape\Large Facultad de Ciencias \par}
%   \vspace{1cm}
%   {\scshape\Large Matemáticas para las Ciencias Aplicadas 3 \par}
%   \vspace{1cm}
%   {\scshape\Large Licenciatura en Ciencias de la Computación \par}
%   \vspace{1cm}
%   {\scshape\Huge 2da lista de problemas  \par}
%   \vspace{2cm}
%   {\itshape\Large Segundo Parcial \par}
%   \vspace{2cm}
%   {\Large Autores: \par}
%   {\Large Ramírez Mendoza Joaquín Rodrigo \par}
%   {\Large Treviño Puebla Héctor Jerome \par}
%   {\Large Villalobos Juarez Gontrán Eliut\par}
%   \vspace{3cm}
%   {\Large Noviembre 2025 \par}
% \end{titlepage}
% ---Fin de la portada de la portada
% \maketitle
% Introducir aquí sus capítulos
\begin{enumerate}[label=\textcolor{blue}{\large\textbf{\arabic*.}}]
  \item [1.] Sean $\mathbf{r} = x\,\mathbf{i} + y\,\mathbf{j} + z\,\mathbf{k}$, $r = \|\mathbf{r}\|$, $f(r)$ y $\mathbf{F}(r) = f(r)\,\mathbf{r}$. Use las propiedades adecuadas enunciadas en los teoremas$^2$ de la página 13 para demostrar que:
\begin{align}
    \nabla f(r) &= \frac{f'(r)}{r}\,\mathbf{r} \tag{5}
\end{align}
El gradiente de \(f(r)\) es:
\[
\nabla f(r) = \frac{\partial f}{\partial x} \mathbf{i} + \frac{\partial f}{\partial y} \mathbf{j} + \frac{\partial f}{\partial z} \mathbf{k}.
\]

Por regla de la cadena:
\[
\frac{\partial f}{\partial x} = f'(r) \frac{\partial r}{\partial x}, \quad
\frac{\partial f}{\partial y} = f'(r) \frac{\partial r}{\partial y}, \quad
\frac{\partial f}{\partial z} = f'(r) \frac{\partial r}{\partial z}.
\]

Calculamos \(\frac{\partial r}{\partial x}\):
\[
r = (x^2 + y^2 + z^2)^{1/2} \Rightarrow \frac{\partial r}{\partial x} = \frac{1}{2}(x^2 + y^2 + z^2)^{-1/2} \cdot 2x = \frac{x}{r}.
\]

\[
\frac{\partial r}{\partial y} = \frac{y}{r}, \quad \frac{\partial r}{\partial z} = \frac{z}{r}.
\]

Sustituimos:
\[
\frac{\partial f}{\partial x} = f'(r) \frac{x}{r}, \quad
\frac{\partial f}{\partial y} = f'(r) \frac{y}{r}, \quad
\frac{\partial f}{\partial z} = f'(r) \frac{z}{r}.
\]

Por lo tanto:
\[
\nabla f(r) = f'(r) \frac{x}{r} \mathbf{i} + f'(r) \frac{y}{r} \mathbf{j} + f'(r) \frac{z}{r} \mathbf{k} = \frac{f'(r)}{r} (x\mathbf{i} + y\mathbf{j} + z\mathbf{k}) = \frac{f'(r)}{r} \mathbf{r}.
\]



\begin{align*}
    \operatorname{div}\,\mathbf{F} &= 3f(r) + r f'(r) \tag{6}
\end{align*}
\[\mathbf{F} = f(r) \mathbf{r} = f(r)x \mathbf{i} + f(r)y \mathbf{j} + f(r)z \mathbf{k}\].

La divergencia es:
\[
\text{div } \mathbf{F} = \frac{\partial}{\partial x}[f(r)x] + \frac{\partial}{\partial y}[f(r)y] + \frac{\partial}{\partial z}[f(r)z].
\]

Calculamos usando la regla del producto:

Para \(\frac{\partial}{\partial x}[f(r)x]\):
\[
\frac{\partial}{\partial x}[f(r)x] = \frac{\partial f(r)}{\partial x} \cdot x + f(r) \cdot \frac{\partial x}{\partial x} = f'(r) \frac{\partial r}{\partial x} \cdot x + f(r) \cdot 1 = f'(r) \frac{x}{r} \cdot x + f(r) = f'(r) \frac{x^2}{r} + f(r).
\]

Entonces:
\[
\frac{\partial}{\partial y}[f(r)y] = f'(r) \frac{y}{r} \cdot y + f(r) = f'(r) \frac{y^2}{r} + f(r),
\]
\[
\frac{\partial}{\partial z}[f(r)z] = f'(r) \frac{z}{r} \cdot z + f(r) = f'(r) \frac{z^2}{r} + f(r).
\]

Sumamos los términos:
\[
\text{div } \mathbf{F} = \left(f'(r) \frac{x^2}{r} + f(r)\right) + \left(f'(r) \frac{y^2}{r} + f(r)\right) + \left(f'(r) \frac{z^2}{r} + f(r)\right)
\]
\[
= 3f(r) + f'(r) \frac{x^2 + y^2 + z^2}{r} = 3f(r) + f'(r) \frac{r^2}{r} = 3f(r) + r f'(r).
\]



\begin{align*}
    \operatorname{rot}\,\mathbf{F} &= \mathbf{0} \tag{7}
\end{align*}
Por la identidad vectorial:
\[
\nabla \times (\phi \mathbf{A}) = \phi (\nabla \times \mathbf{A}) - \mathbf{A} \times (\nabla \phi),
\]
con \(\phi = f(r)\) y \(\mathbf{A} = \mathbf{r}\).

Primero, \(\nabla \times \mathbf{r} = \mathbf{0}\) porque:
\[
\nabla \times \mathbf{r} = 
\begin{vmatrix}
\mathbf{i} & \mathbf{j} & \mathbf{k} \\
\frac{\partial}{\partial x} & \frac{\partial}{\partial y} & \frac{\partial}{\partial z} \\
x & y & z
\end{vmatrix}
= \left(\frac{\partial z}{\partial y} - \frac{\partial y}{\partial z}\right)\mathbf{i} - \left(\frac{\partial z}{\partial x} - \frac{\partial x}{\partial z}\right)\mathbf{j} + \left(\frac{\partial y}{\partial x} - \frac{\partial x}{\partial y}\right)\mathbf{k} = \mathbf{0}.
\]

Segundo, \(\nabla \phi = \nabla f(r) = \frac{f'(r)}{r} \mathbf{r}\) por el inciso 1.

Tercero, \(\mathbf{A} \times (\nabla \phi) = \mathbf{r} \times \left(\frac{f'(r)}{r} \mathbf{r}\right) = \frac{f'(r)}{r} (\mathbf{r} \times \mathbf{r}) = \mathbf{0}\).

Sustituimos:
\[
\nabla \times \mathbf{F} = f(r) (\nabla \times \mathbf{r}) - \mathbf{r} \times (\nabla f(r)) = f(r) \cdot \mathbf{0} - \mathbf{0} = \mathbf{0}.
\]




\begin{align*}
    \nabla^2 f(r) &= 2\,\frac{f'(r)}{r} + f''(r) \tag{8}
\end{align*}

Entonces \(f(r)\) es:
\[
\nabla^2 f(r) = \nabla \cdot (\nabla f(r)) = \nabla \cdot \left(\frac{f'(r)}{r} \mathbf{r}\right).
\]

Usamos el resultado del inciso 2, entonces:
\[
\nabla \cdot (g(r) \mathbf{r}) = 3g(r) + r g'(r).
\]

Calculamos \(g'(r)\):
\[
g(r) = \frac{f'(r)}{r} \Rightarrow g'(r) = \frac{d}{dr}\left(\frac{f'(r)}{r}\right) = \frac{f''(r) \cdot r - f'(r) \cdot 1}{r^2} = \frac{r f''(r) - f'(r)}{r^2}.
\]

Entonces:
\[
\nabla^2 f(r) = 3 \cdot \frac{f'(r)}{r} + r \cdot \frac{r f''(r) - f'(r)}{r^2} = \frac{3f'(r)}{r} + \frac{r f''(r) - f'(r)}{r}
\]
\[
= \frac{3f'(r) + r f''(r) - f'(r)}{r} = \frac{2f'(r) + r f''(r)}{r} = 2 \frac{f'(r)}{r} + f''(r).
\]
\newpage
  \item [2.] Use los resultados adecuados del problema anterior para demostrar que la divergencia de un campo cuadrado-inverso es cero.\newpage
  \item [3.] Use el resultado adecuado del primer problema para demostrar que si $\mathbf{F}$ es un campo vectorial de la forma $\mathbf{F}(r) = f(\|\mathbf{r}\|)\,\mathbf{r}$ y su divergencia $\text{div}\,\mathbf{F} = 0$, entonces $\mathbf{F}$ es un campo cuadrado-inverso. (Sugerencia: sea $r = \|\mathbf{r}\|$, multiplique por $r^2$ ambos miembros de la ecuación y resuelva la ecuación diferencial resultante para demostrar que $f(r)$ es de tipo cuadrado-inverso).

Partimos de que $\text{div}\,\mathbf{F} =0$

Entonces, por el primer ejercicio, la ecuación (6) tenemos que $3f(r) + rf'(r)= 0$

Multiplicando por $r^2$ ambos miembros de la ecuación, obtenemos:
\begin{align*}
    r^2 \cdot\left[3f(r) + r f'(r)\right] & = 0 \\
    r^2 \cdot 3f(r) + r^3 f'(r)           & = 0 \\
    3r^2 f(r) + r^3 f'(r)                 & = 0
\end{align*}

Notemos que la expresión del lado izquierdo es la derivada del producto $r^3 f(r)$, es decir:
\begin{align*}
    \frac{d}{dr}\left[r^3 f(r)\right] & =\left(\frac{d}{dr}r^3\right)f(r) + r^3 \left(\frac{d}{dr}f(r)\right)       \\
    \frac{d}{dr}\left[r^3 f(r)\right] & =3r^2 f(r) + r^3 f'(r)\\
    \text{Esto implica que }\frac{d}{dr}\left[r^3 f(r)\right] & = 0
\end{align*}

Por el Teorema Fundamental del Cáculo, la integral de la derivada es igual a la función más una constante, es decir:
\begin{align*}
    \int \frac{d}{dr}\left[r^3 f(r)\right] dr & = \int 0 \, dr \\
    r^3 f(r) & = C
\end{align*}

Finalmente, despejando $f(r)$, obtenemos:
\begin{align*}
    f(r) & = \frac{C}{r^3}
\end{align*}

Sustituyendo $f(r)$ en la expresión de $\mathbf{F}(r)$, tenemos:
\begin{align*}
    \mathbf{F}(r) & = f(\|\mathbf{r}\|)\,\mathbf{r} \\
    \mathbf{F}(r) & = \frac{C}{\|\mathbf{r}\|^3}\,\mathbf{r}\\
\end{align*}



Por lo tanto, hemos demostrado que si la divergencia de $\mathbf{F}$ es cero, entonces $\mathbf{F}$ es un campo cuadrado-inverso que por definición es de la forma $\mathbf{F}(r) = \frac{C}{\|\mathbf{r}\|^3}\,\mathbf{r}$\newpage
\end{enumerate}
\subsection*{la integral de linea}
\begin{enumerate}
  \item [4.] Mostrar que la masa $M$ de un alambre de grosor desdeñable cuya forma es la de la curva parametrizada por $x = 2t,\, y = \ln t,\,z = 4\sqrt{t}$ para $1 \le t \le 4$ y cuya densidad $\delta(x, y, z)$ es proporcional a la distancia del punto $(x, y, z)$ al plano $xy$ con constante de proporcionalidad $k$ es
$$M = \frac{136k}{3} \text{ unidades de masa.}$$

La masa $M$ de un alambre de grosor desdeñable que ocupa una curva $C$ con función de densidad $\delta(x,y,z)$ se calcula mediante la fórmula de la integral de línea
$$\int_C \delta(x,y,z)ds$$

Tenemos que $\delta(x,y,z)$ es proporcional a la distancia del punto $(x,y,z)$ al plano $xy$ con una constante de proporcionalidad $k$.

La distancia de un punto $(x,y,z)$ al plano $xy$ es simplementen $z$ pues es siempre positiva $(z\ge 4)$. Entonces $\delta(x,y,z) = kz$

$$\delta(z(t)) = k \left(4\sqrt{t}\right) = 4k t^{1/2}$$

Calculando la diferencial de arco $\displaystyle ds = \sqrt{\left( \frac{dx}{dt}\right)^2 + \left(\frac{dy}{dt}\right)^2}dt$

Calculando al derivada de cada componete tenemos que: 

\begin{align*}
    \frac{dx}{dt} = \frac{d}{dt}(2t) = 2, \quad \frac{dy}{dt} = \frac{d}{dt}(\ln t) = \frac{1}{t}, \quad \frac{dz}{dt} = \frac{d}{dt}(4\sqrt{t}) = \frac{2}{\sqrt{t}}
\end{align*}

Sustituyendo en la fórmula de $ds$ tenemos que:

\begin{align*}
    ds &= \sqrt{(2)^2 + \left(\frac{1}{t}\right)^2 + \left(\frac{2}{\sqrt{t}}\right)^2} dt \\
    &= \sqrt{4 + \frac{1}{t^2} + \frac{4}{t}} dt \\
    &= \sqrt{\frac{4t^2 + 4t + 1}{t^2}} dt \\
    &= \frac{\sqrt{(2t + 1)^2}}{t} dt \\
    &= \frac{2t + 1}{t} dt
\end{align*}

Entonces la masa $M$ se calcula como:

\begin{align*}
    M &= \int_C \delta(x,y,z) ds = \int_1^4 4k t^{1/2} \cdot \frac{2t + 1}{t} dt = 4k \int_1^4 t^{1/2} \left(2 + \frac{1}{t}\right) dt \\
    &= 4k \int_1^4 \left(2t^{1/2} + t^{-1/2}\right) dt = 4k \left[ \frac{4}{3} t^{3/2} + 2 t^{1/2} \right]_1^4 = 4k \left( \left[ \frac{4}{3} (4)^{3/2} + 2 (4)^{1/2} \right] - \left[ \frac{4}{3} (1)^{3/2} + 2 (1)^{1/2} \right] \right) \\
    &= 4k \left( \left[ \frac{4}{3} (8) + 2 (2) \right] - \left[ \frac{4}{3} (1) + 2 (1) \right] \right) = 4k \left( \left[ \frac{32}{3} + 4 \right] - \left[ \frac{4}{3} + 2 \right] \right) \\
    &= 4k \left( \frac{32}{3} + 4 - \frac{4}{3} - 2 \right) \\
    &= 4k \left( \frac{28}{3} + 2 \right) \\
    &= 4k \cdot \frac{34}{3} = \frac{136k}{3}
\end{align*}    \newpage
  \item [5.] Un granjero que pesa $150 \text{ lb}$ sube un saco de grano que pesa $20 \text{ lb}$ a lo largo de una escalera helicoidal alrededor de un silo cilíndrico de radio $25 \text{ ft}$. Conforme el granjero sube, el grano se sale del saco a razón de $1 \text{ lb}$ por cada $10 \text{ ft}$ de ascenso. ¿Cuánto trabajo realiza el granjero si asciende verticalmente $60 \text{ ft}$ en exactamente cuatro vueltas de la hélice? (\textbf{Sugerencia}: describa como un campo vectorial la fuerza ejercida por el granjero para levantar su propio peso más el peso del saco en cada punto de su trayectoria).\\

La componente vertical de la fuerza es:
\[
F_z = 150 + \left(20 - \frac{z}{10}\right) = 170 - \frac{z}{10} \quad \text{lb}.
\]

El trabajo realizado es la integral de \(F_z\) con respecto a \(z\) desde 0 hasta 60 ft:
\[
W = \int_{0}^{60} \left(170 - \frac{z}{10}\right) dz.
\]

\[
W = \left[ 170z - \frac{z^2}{20} \right]_{0}^{60} = \left(170 \cdot 60 - \frac{60^2}{20}\right) - (0) = 10200 - \frac{3600}{20} = 10200 - 180 = 10020 \quad \text{lb-ft}.
\]

El trabajo del granjero es \(10020\) lb-ft.\newpage
  \item [6.] Supóngase que una partícula se mueve a través del campo de fuerzas
$$\mathbf{F}(x, y) = xy \mathbf{i} + (x - y) \mathbf{j}$$
desde el punto $(0, 0)$ hasta el punto $(1, 0)$ a lo largo de la curva $x = t, y = \lambda t(1 - t)$. ¿Para qué valor de $\lambda$ el trabajo hecho por el campo de fuerzas será $1$ unidad de trabajo?\newpage
  \item [7.] Considere la lata cilíndrica cortada que se muestra en la figura 15.
\begin{enumerate}
  \item Dé un argumento geométrico simple para explicar por qué el área $A$ de la pared lateral de la lata es $A = 4\pi \text{ unidades de área}$.
  \item Describa el área de la pared lateral de la lata mediante una integral de línea.
  \item Use la integral de línea para corroborar que $A = 4\pi \text{ unidades de área}$.
\end{enumerate}

\begin{center}
  \includegraphics[height = 0.14\textheight]{recursos/image.png}\par
\end{center}
\textbf{Figura 15:} La curva sinusoidal $z(t) = 2 + 0.5 \sin 3t$ corta la pared lateral del cilindro $x^2 + y^2 = 1$.

(a) Vemos que la base de la lata es el círculo unitario $x^2 + y^2 = 1$ en el plano $xy$ cuyo perímetro es $P = 2 \pi r = 2 \pi (1) = 2 \pi$ unidades de longitud.
La altura de la lata varía según la función sinusoidal $z(t) = 2 + 0.5 \sin 3t$ que oscila entre $1.5$ y $2.5$ unidades de longitud.

Podemos decir que por simetría, la altura promedio de la pared lateral de la lata es el valor medio de la función sinusoidal en un período completo. En este caso, el intervalo es $0\le t\le 2\pi$ dond en $\sin3t$ completa $3$ ciclos completos.

Por lo tanto, la altura promedio de la pared lateral de la lata es $2$

Finalmente, el área de la pared lateral de la lata es el producto del perímetro de la base por la altura promedio de la pared lateral. $$A=2\pi\cdot2=4\pi$$

(b) El área de la pared lateral de la lata es una integral de línea de un campo escalar respecto a la longitud de arco. La función escalar $f(x,y,z)$ es la proyección de la base de la lata en el plano $xy$ que es el círculo unitario.

El círculo $x^2+y^2=1$ en el plano $xy$ se parametriza como: $$\mathbf{r}(t) = \langle \cos t, \sin t \rangle, \quad 0 \le t \le 2\pi$$

Para cada altura $z = z(t) = 2 + 0.5 \sin 3t$, la función escalar es $f(x,y,z) = z(t)$.
Entonces, si $f(x,y)\ge 0$, la integral de linea que describe el área de la pared lateral de la lata es:
$$A = \oint_C f(x,y) ds = \int_0^{2\pi} (2 + 0.5 \sin 3t)  ds$$

(c) Calculamos la diferencial de arco $ds$ como:
\begin{align*}
  ds &= \sqrt{\left(\frac{dx}{dt}\right)^2 + \left(\frac{dy}{dt}\right)^2} dt \\
  &= \sqrt{(-\sin t)^2 + (\cos t)^2} dt \\
  &= \sqrt{\sin^2 t + \cos^2 t} dt \\
  &= \sqrt{1} dt = dt
\end{align*}

Sustituyendo en la integral de línea tenemos:
\begin{align*}
  A &= \int_0^{2\pi} (2 + 0.5 \sin 3t) dt \\
  &= \left[ 2t - \frac{0.5}{3} \cos 3t \right]_0^{2\pi} \\
  &= \left( 2(2\pi) - \frac{0.5}{3} \cos(6\pi) \right) - \left( 2(0) - \frac{0.5}{3} \cos(0) \right) \\
  &= \left( 4\pi - \frac{0.5}{3}(1) \right) - \left( 0 - \frac{0.5}{3}(1) \right) \\
  &= 4\pi - \frac{0.5}{3} + \frac{0.5}{3} = 4\pi
\end{align*}

Esto confirma que el área de la pared lateral de la lata es $A = 4\pi$ unidades de área.\newpage
\end{enumerate}


\begin{enumerate}
  \subsection*{Problema (Independencia de la Trayectoria)}
  \item [8.]   Sea $\mathbf{F}(x, y) = \frac{y}{x^2 + y^2} \mathbf{i} - \frac{x}{x^2 + y^2} \mathbf{j}$.
    \begin{enumerate}
        \item[(a)] Demostrar que si $C_1$ es el arco circular superior del círculo $x^2 + y^2 = 1$ que va de $P_1 = (1, 0)$ a $P_2 = (-1, 0)$ y $C_2$ es el arco inferior del mismo círculo que va de $P_1$ a $P_2$, entonces:
        $$\int_{C_1} \mathbf{F} \cdot d\mathbf{r} \ne \int_{C_2} \mathbf{F} \cdot d\mathbf{r}$$
        \text{Para $C_1$ (arco superior):}
        \begin{align*}
        \text{Parametrización: } & x = \cos t, \quad y = \sin t, \quad t \in [0, \pi] \\
        d\mathbf{r} & = (-\sin t, \cos t) dt \\
        \mathbf{F} & = \frac{\sin t}{\cos^2 t + \sin^2 t} \mathbf{i} - \frac{\cos t}{\cos^2 t + \sin^2 t} \mathbf{j} = \sin t \, \mathbf{i} - \cos t \, \mathbf{j} \\
        \mathbf{F} \cdot d\mathbf{r} & = (\sin t)(-\sin t) + (-\cos t)(\cos t) dt \\
        & = -\sin^2 t - \cos^2 t dt = -(\sin^2 t + \cos^2 t) dt = -dt \\
        \int_{C_1} \mathbf{F} \cdot d\mathbf{r} & = \int_{0}^{\pi} -dt = -[t]_{0}^{\pi} = -\pi
        \end{align*}
        
        \text{Para $C_2$ (arco inferior):}
        \begin{align*}
        \text{Parametrización: } & x = \cos t, \quad y = \sin t, \quad t \in [0, -\pi] \\
        d\mathbf{r} & = (-\sin t, \cos t) dt \\
        \mathbf{F} & = \sin t \, \mathbf{i} - \cos t \, \mathbf{j} \\
        \mathbf{F} \cdot d\mathbf{r} & = -\sin^2 t - \cos^2 t dt = -dt \\
        \int_{C_2} \mathbf{F} \cdot d\mathbf{r} & = \int_{0}^{-\pi} -dt = -[t]_{0}^{-\pi} = -(-\pi - 0) = \pi 
        \end{align*}
        
        Por lo tanto, $\int_{C_1} \mathbf{F} \cdot d\mathbf{r} = -\pi \neq \pi = \int_{C_2} \mathbf{F} \cdot d\mathbf{r}$. \\

        
        \item[(b)] Probar que si $f(x, y) = \frac{y}{x^2 + y^2}$ y $g(x, y) = -\frac{x}{x^2 + y^2}$, entonces:
        $$\frac{\partial g}{\partial x} = \frac{\partial f}{\partial y}$$

        
           
        \[\frac{\partial f}{\partial y}  = \frac{\partial}{\partial y}\left(\frac{y}{x^2 + y^2}\right) \]
           \[  = \frac{1 \cdot (x^2 + y^2) - y \cdot 2y}{(x^2 + y^2)^2} \]
           \[  = \frac{x^2 + y^2 - 2y^2}{(x^2 + y^2)^2} = \frac{x^2 - y^2}{(x^2 + y^2)^2} \]
            
           \[ \frac{\partial g}{\partial x}  = \frac{\partial}{\partial x}\left(-\frac{x}{x^2 + y^2}\right) \]
            \[ = -\frac{1 \cdot (x^2 + y^2) - x \cdot 2x}{(x^2 + y^2)^2} \]
            \[ = -\frac{x^2 + y^2 - 2x^2}{(x^2 + y^2)^2} = -\frac{-x^2 + y^2}{(x^2 + y^2)^2} = \frac{x^2 - y^2}{(x^2 + y^2)^2}\]
            
            
            Por lo tanto, \[\frac{\partial g}{\partial x} = \frac{\partial f}{\partial y}\].


        \item[(c)] Explique por qué los resultados de los dos incisos anteriores no contradicen el \textbf{criterio de conservatividad}.\\
        \item[] Aunque se cumple que $\dfrac{\partial g}{\partial x} = \dfrac{\partial f}{\partial y}$, esto no implica que $F$ sea conservativo, pues el \text{criterio de conservatividad} solo es válido en dominios simplemente conexos. En este caso, $F$ no está definida en el origen, por lo que su dominio $R^2 \setminus \{(0,0)\}$ no es simplemente conexo. Como consecuencia, la integral de línea de $F$ depende de la trayectoria, tal como se comprobó en el inciso (a).

    \end{enumerate}\newpage
  \item [9.]  Sea $\mathbf{F} = \nabla \phi$, donde $\phi(x,y) = \sin (x - 2y)$. Hallar sendas curvas planas $C_1$ y $C_2$ tales que
\begin{enumerate}
    \item[(a)] $\int_{C_1} \mathbf{F} \cdot d\mathbf{r} = 0$
    \item[(b)] $\int_{C_2} \mathbf{F} \cdot d\mathbf{r} = 1$
\end{enumerate}

El Teorema Fundamental de Integrales de Línea establece que si $\mathbf{F} = \nabla \phi$ es un campo conservativo en algunda región abierta $D\subset \mathbb{R}^2$ que contiene los puntos $(x_0,y_0)$ y $(x_1,y_1)$ y que $f(x,y)$ y $g(x,y)$ son continuas en $D$ y $C$ cualquier curva parametrizada suave por pedazos que empieze en $(x_0,y_0)$ y termine en $(x_1,y_1)$ y esté contenida en la región $D$, entonces:
$$\int_C\mathbf{F}(x,y)\cdot d\mathbf{r} = \phi(x_1,y_1) - \phi(x_0,y_0)$$.

(a) Para que la integral de línea sea cero, necesitamos que $\phi(x_1,y_1) = \phi(x_0,y_0)$. Podemos elegir cualquier par de puntos que satisfagan esta condición. Por ejemplo, podemos elegir los puntos $(0,0)$ y $(2\pi, \pi)$, ya que:
$$\phi(0,0) = \sin(0 - 0) = 0$$
$$\phi(2\pi, \pi) = \sin(2\pi - 2\pi) = \sin(0) = 0$$
Por lo tanto, cualquier curva $C_1$ que conecte estos dos puntos servirá. Una posible elección es la línea recta parametrizada por:
$$\mathbf{r}_1(t) = (2\pi t, \pi t), \quad t \in [0,1]$$.

Otra posible elección son las curvas tales que empiezan y terminan en el mismo punto, es decir curvas cerradas, por ejemplo la curva que empieza y termina en el punto $(0,0)$.

entonces $\displaystyle \int_{C_1} \mathbf{F} \cdot d\mathbf{r} = \phi(0,0) - \phi(0,0) = 0$.

(b) Para que la integral de línea sea igual a 1, necesitamos encontrar puntos $(x_0,y_0)$ y $(x_1,y_1)$ tales que $\phi(x_1,y_1) - \phi(x_0,y_0) = 1$. 

Como $\phi(x,y) = \sin(x - 2y)$, la condición anterior se cumple si y solo si el potencial en el punto final es 1 unidad mayor que en el punto inicial pues el rango de los valores de $\phi$ es $\left[-1,1\right]$. Entonces debe pasar que $\phi(x_0,y_0) = 0$ y $\phi(x_1,y_1) = 1$.

Sea $x-2y = 0$, ya vimos que para $(x_0,y,0) = (0,0)$, $\phi(0,0) = 0$. Ahora buscamos un punto $(x_1,y_1)$ tal que $x_1 - 2y_1 = \frac{\pi}{2}$, por ejemplo $(\frac{\pi}{2},0)$, ya que:
$$\phi\left(\frac{\pi}{2},0\right) = \sin\left(\frac{\pi}{2} - 0\right) = \sin\left(\frac{\pi}{2}\right) = 1$$.

Podemos concluir que cualquier curva suave por pedazos $C_2$ que conecte los puntos $(0,0)$ y $\left(\frac{\pi}{2},0\right)$ servirá. Una posible elección es la línea recta parametrizada por:
$$\mathbf{r}_2(t) = \left(\frac{\pi}{2} t, 0\right), \quad t \in [0,1]$$.

Entonces, $\displaystyle \int_{C_2} \mathbf{F} \cdot d\mathbf{r} = \phi\left(\frac{\pi}{2},0\right) - \phi(0,0) = 1 - 0 = 1$.\newpage
  \item [10.] Recuerde que si $\phi = -\frac{c}{\|\mathbf{r}\|}$ y $\mathbf{F} = \frac{c}{\|\mathbf{r}\|^3} \mathbf{r}$ es un campo cuadrado-inverso, entonces $\nabla \phi = \mathbf{F}$.

\begin{enumerate}
    \item[(a)] Sean $P_1, P_2 \in \mathbb{R}^3$ dos puntos distintos del origen cuyas distancias a éste son, respectivamente, $d_1$ y $d_2$. Mostrar que el trabajo $W$ que hace $\mathbf{F}$ para desplazar una partícula desde $P_1$ hasta $P_2$ es
        $$W = c \left( \frac{1}{d_1} - \frac{1}{d_2} \right).$$

        Para obtener $W$ necesitamos, W = $\displaystyle\int_C\mathbf{F}d\mathbf{r}$ pero dado que tenemos $\phi$ tal que $\nabla \phi = \mathbf{F}$, podemos usar el Teorema Fundamental de las integrales de línea: \[\int_C \mathbf{F}d\mathbf{r} = \phi(x_1,y_1) - \phi(x_0,y_0)\]

        Así $W = \phi(x_1,y_1) - \phi(x_0,y_0)$, tomaremos $(x_0,y_0 = P_1 \text{ y } (x_1,y_1) = P_2$

        Así, tenemos:\[W = -\frac{c}{||\mathbf{r}_2||}-\left(-\frac{c}{||\mathbf{r}_1||}\right)\]

        Ahora para el valor de $||\mathbf{r}_1||$ y $||\mathbf{r}_2||$, dado que son radio-vectores, estas magnitudes son la distancia del origen al punto, que nos son dadas en el problema como $d_1$ y $d_2$ respectivamente.

        Así
        \begin{align*}
            W & = -\frac{c}{d_2} - \left(-\frac{c}{d_1} \right) \\
              & = -\frac{c}{d_2} + \frac{c}{d_1}                \\
              & = c\left(\frac{1}{d_1} - \frac{1}{d_2}\right)   \\
            W & = c\left(\frac{1}{d_1} - \frac{1}{d_2}\right)
        \end{align*}

    \item[(b)] Considere la acción del campo gravitacional del Sol sobre la Tierra:
        $$\mathbf{F} = -G \frac{mM}{\|\mathbf{r}\|^3} \mathbf{r}$$
        donde $G = 6.67 \times 10^{-11} \frac{\text{N} \cdot \text{m}^2}{\text{kg}^2}$, $m = 5.97 \times 10^{24} \text{ kg}$ y $M = 1.99 \times 10^{30} \text{ kg}$. Use el resultado del inciso anterior para calcular el trabajo que hace $\mathbf{F}$ para desplazar a la Tierra desde su afelio ($a \text{ } 1.52 \times 10^8 \text{ km}$ de distancia del Sol) hasta su perihelio ($a \text{ } 1.47 \times 10^8 \text{ km}$ de distancia del Sol).

        Notemos que el sol es el oringe.

        Usaremos \[W = c\left(\frac{1}{d_1} - \frac{1}{d_2}\right)\]
        con
        \begin{align*}
            c & = \left(-1\right)\left(6.67 \times 10^{-11}\right)\left(5.97 \times 10^{24}\right)\left(1.99 \times 10^{30}\right) = -7.92729766 \times 10^{44} \text{ N} \cdot \text{m}^2 \\
            d_1 & = 1.52 \times 10^8 \text{ km} = 1.52 \times 10^{11} \text{ m}                                                                                 \\
            d_2 & = 1.47 \times 10^8 \text{ km} = 1.47 \times 10^{11} \text{ m}
        \end{align*}

        Así resulta
        \begin{align*}
            W &= -7.92729766 \times 10^{44} \left(\frac{1}{1.52 \times 10^{11}} - \frac{1}{1.47 \times 10^{11}}\right) \\
              & = -7.92729766 \times 10^{44} \left(6.57894737 \times 10^{-12} - 6.80272109 \times 10^{-12}\right)        \\
              & = -7.92729766 \times 10^{44} \left(-2.23873719 \times 10^{-13}\right)                                   \\
            W & = 1.77437314 \times 10^{32} \text{ J}
        \end{align*}
\end{enumerate}
\newpage


        \subsection*{Problemas (Cálculo del área, Teorema de Green)}
  \item [11.]\begin{enumerate}
        \item[(a)] Si $C$ es el segmento de recta que une $(x_1, y_1)$ con $(x_2, y_2)$, probar que:
        $$\int_C x\,dy - y\,dx = x_1 y_2 - x_2 y_1.$$

        Parametrización del segmento de $(x_1, y_1)$ a $(x_2, y_2)$:
        \begin{align*}
        x(t) & = x_1 + t(x_2 - x_1), \quad y(t) = y_1 + t(y_2 - y_1), \quad t \in [0,1] \\
        dx & = (x_2 - x_1) dt, \quad dy = (y_2 - y_1) dt
        \end{align*}
        
        \begin{align*}
        x \, dy - y \, dx & = [x_1 + t(x_2 - x_1)](y_2 - y_1) dt - [y_1 + t(y_2 - y_1)](x_2 - x_1) dt \\
        & = [x_1(y_2 - y_1) + t(x_2 - x_1)(y_2 - y_1) - y_1(x_2 - x_1) - t(y_2 - y_1)(x_2 - x_1)] dt \\
        & = [x_1(y_2 - y_1) - y_1(x_2 - x_1)] dt \\
        & = (x_1 y_2 - x_1 y_1 - y_1 x_2 + y_1 x_1) dt = (x_1 y_2 - x_2 y_1) dt
        \end{align*}
        
        \begin{align*}
        \int_C x \, dy - y \, dx & = \int_0^1 (x_1 y_2 - x_2 y_1) dt = (x_1 y_2 - x_2 y_1)[t]_0^1 = x_1 y_2 - x_2 y_1
        \end{align*}

        \item[(b)] Si los vértices de un polígono (sus aristas no se intersectan) en orden levógiro son $(x_1, y_1), (x_2, y_2), \dots, (x_n, y_n)$, demostrar que el área $A$ del polígono es:
        
        Para un polígono con vértices $(x_1, y_1), (x_2, y_2), \ldots, (x_n, y_n)$:
        
        \[A  = \frac{1}{2} \oint_{\partial D} (x \, dy - y \, dx) \]
        \[ = \frac{1}{2} \left[ \sum_{i=1}^{n-1} (x_i y_{i+1} - x_{i+1} y_i) + (x_n y_1 - x_1 y_n) \right] \]
        

        $$A = \frac{1}{2} \left[ (x_1 y_2 - x_2 y_1) + (x_2 y_3 - x_3 y_2) + \dots + (x_{n-1} y_n - x_n y_{n-1}) + (x_n y_1 - x_1 y_n) \right].$$

        \item[(c)] Hallar el área del pentágono con vértices en $(0, 0)$, $(2, 1)$, $(1, 3)$, $(0, 2)$ y $(-1, 1)$.
        

    
    \[A  = \frac{1}{2} \left[ (0 \cdot 1 - 2 \cdot 0) + (2 \cdot 3 - 1 \cdot 1) + (1 \cdot 2 - 0 \cdot 3) + (0 \cdot 1 - (-1) \cdot 2) + ((-1) \cdot 0 - 0 \cdot 1) \right] \]
     \[= \frac{1}{2} \left[ 0 + (6 - 1) + (2 - 0) + (0 + 2) + 0 \right] \]
     \[= \frac{1}{2} (0 + 5 + 2 + 2 + 0) = \frac{1}{2} \cdot 9 = 4.5 \]
    
    
    El área del pentágono es 4.5 unidades cuadradas.
    \end{enumerate}\newpage
  \item[12.]  Usar una integral de línea para calcular el área encerrada por la astroide
    $$x(\phi) = a \cos^3 \phi, \quad y(\phi) = a \sin^3 \phi, \quad \text{para } 0 \le \phi \le 2\pi.$$
    y dibujar la astroide cuando $a=2$.\newpage
  \item [13.]  Sea $\mathbf{F}(x, y) = (e^{-x} + 3y) \mathbf{i} + x \mathbf{j}$ y sea $C$ la frontera de la región del plano dentro del círculo $x^2 + y^2 = 16$ y fuera del círculo $x^2 - 2x + y^2 = 3$. Usar el \textbf{teorema de Green generalizado} para calcular $\oint_C \mathbf{F} \cdot d\mathbf{r}$.


El campo vectorial es de la forma $\mathbf{F}(x, y) = f(x, y) \mathbf{i} + g(x, y) \mathbf{j}$:
$$f(x, y) = e^{-x} + 3y$$
$$g(x, y) = x$$

La curva $C$ es la frontera de la región $D$ definida por:

La frontera exterior $C_1$ dentro del círculo $x^2 + y^2 = 16$ es un círculo centrado en $(0, 0)$ con radio $r=4$.

La frontera interior $C_2$ fuera del círculo $x^2 - 2x + y^2 = 3$, donde, al completar el cuadrado tenemos la ecuación $(x - 1)^2 + y^2 = 4$. Este es un círculo centrado en $(1, 0)$ con radio $r=2$.

La región $D$ es, por lo tanto, el área entre los círculos concentricos $C_1$ y $C_2$. Como $D$ contiene un "hoyo", es una región múltiplemente conexa.

El Teorema de Green establece que, para una región $D$ acotada por una curva cerrada $C$ orientada positivamente (en sentido contrario a las manecillas del reloj), la integral de línea es igual a una integral doble sobre $D$:

$$\oint_C \mathbf{F} \cdot d\mathbf{r} = \iint_D \left( \frac{\partial g}{\partial x} - \frac{\partial f}{\partial y} \right) dA$$

Para una región múltiplemente conexa como $D$, la frontera $C$ se orienta positivamente si la curva exterior $C_1$ se recorre en contra de las manecillas del reloj y la curva interior $C_2$ se recorre en sentido a favor de las manecillas del reloj, de modo que la región $D$ siempre quede a la izquierda.


Calculamos las derivadas parciales de $f$ y $g$:

$$\frac{\partial f}{\partial y} = \frac{\partial}{\partial y} (e^{-x} + 3y) = 3$$
$$\frac{\partial g}{\partial x} = \frac{\partial}{\partial x} (x) = 1$$

El integrando de Green es:
$$\frac{\partial g}{\partial x} - \frac{\partial f}{\partial y} = 1 - 3 = -2$$

La integral de línea se convierte en:

$$\oint_C \mathbf{F} \cdot d\mathbf{r} = \iint_D (-2)\, dA = -2 \iint_D \, dA $$

Notemos que el área es el área del círculo grande $C_1$ menos el área del círculo pequeño $C_2$, por lo que 

$$\oint_C \mathbf{F} \cdot d\mathbf{r} = -2 \iint_D dA = -2 \left[\pi (4)^2 - \pi(2)^2\right]$$
$$\oint_C \mathbf{F} \cdot d\mathbf{r} = -24\pi$$

Finalmente, nuestro cálculo de $\oint_C \mathbf{F} \cdot d\mathbf{r}$ utilizando el teorema de Green generalizado muestra que el área de la región es $-24\pi$.\newpage
\end{enumerate}

\end{document}



