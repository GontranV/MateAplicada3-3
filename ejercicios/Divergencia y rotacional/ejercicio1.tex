Sean $\mathbf{r} = x\,\mathbf{i} + y\,\mathbf{j} + z\,\mathbf{k}$, $r = \|\mathbf{r}\|$, $f(r)$ y $\mathbf{F}(r) = f(r)\,\mathbf{r}$. Use las propiedades adecuadas enunciadas en los teoremas$^2$ de la página 13 para demostrar que:
\begin{align}
    \nabla f(r) &= \frac{f'(r)}{r}\,\mathbf{r} \tag{5}
\end{align}
El gradiente de \(f(r)\) es:
\[
\nabla f(r) = \frac{\partial f}{\partial x} \mathbf{i} + \frac{\partial f}{\partial y} \mathbf{j} + \frac{\partial f}{\partial z} \mathbf{k}.
\]

Por regla de la cadena:
\[
\frac{\partial f}{\partial x} = f'(r) \frac{\partial r}{\partial x}, \quad
\frac{\partial f}{\partial y} = f'(r) \frac{\partial r}{\partial y}, \quad
\frac{\partial f}{\partial z} = f'(r) \frac{\partial r}{\partial z}.
\]

Calculamos \(\frac{\partial r}{\partial x}\):
\[
r = (x^2 + y^2 + z^2)^{1/2} \Rightarrow \frac{\partial r}{\partial x} = \frac{1}{2}(x^2 + y^2 + z^2)^{-1/2} \cdot 2x = \frac{x}{r}.
\]

\[
\frac{\partial r}{\partial y} = \frac{y}{r}, \quad \frac{\partial r}{\partial z} = \frac{z}{r}.
\]

Sustituimos:
\[
\frac{\partial f}{\partial x} = f'(r) \frac{x}{r}, \quad
\frac{\partial f}{\partial y} = f'(r) \frac{y}{r}, \quad
\frac{\partial f}{\partial z} = f'(r) \frac{z}{r}.
\]

Por lo tanto:
\[
\nabla f(r) = f'(r) \frac{x}{r} \mathbf{i} + f'(r) \frac{y}{r} \mathbf{j} + f'(r) \frac{z}{r} \mathbf{k} = \frac{f'(r)}{r} (x\mathbf{i} + y\mathbf{j} + z\mathbf{k}) = \frac{f'(r)}{r} \mathbf{r}.
\]



\begin{align*}
    \operatorname{div}\,\mathbf{F} &= 3f(r) + r f'(r) \tag{6}
\end{align*}
\[\mathbf{F} = f(r) \mathbf{r} = f(r)x \mathbf{i} + f(r)y \mathbf{j} + f(r)z \mathbf{k}\].

La divergencia es:
\[
\text{div } \mathbf{F} = \frac{\partial}{\partial x}[f(r)x] + \frac{\partial}{\partial y}[f(r)y] + \frac{\partial}{\partial z}[f(r)z].
\]

Calculamos usando la regla del producto:

Para \(\frac{\partial}{\partial x}[f(r)x]\):
\[
\frac{\partial}{\partial x}[f(r)x] = \frac{\partial f(r)}{\partial x} \cdot x + f(r) \cdot \frac{\partial x}{\partial x} = f'(r) \frac{\partial r}{\partial x} \cdot x + f(r) \cdot 1 = f'(r) \frac{x}{r} \cdot x + f(r) = f'(r) \frac{x^2}{r} + f(r).
\]

Entonces:
\[
\frac{\partial}{\partial y}[f(r)y] = f'(r) \frac{y}{r} \cdot y + f(r) = f'(r) \frac{y^2}{r} + f(r),
\]
\[
\frac{\partial}{\partial z}[f(r)z] = f'(r) \frac{z}{r} \cdot z + f(r) = f'(r) \frac{z^2}{r} + f(r).
\]

Sumamos los términos:
\[
\text{div } \mathbf{F} = \left(f'(r) \frac{x^2}{r} + f(r)\right) + \left(f'(r) \frac{y^2}{r} + f(r)\right) + \left(f'(r) \frac{z^2}{r} + f(r)\right)
\]
\[
= 3f(r) + f'(r) \frac{x^2 + y^2 + z^2}{r} = 3f(r) + f'(r) \frac{r^2}{r} = 3f(r) + r f'(r).
\]



\begin{align*}
    \operatorname{rot}\,\mathbf{F} &= \mathbf{0} \tag{7}
\end{align*}
Por la identidad vectorial:
\[
\nabla \times (\phi \mathbf{A}) = \phi (\nabla \times \mathbf{A}) - \mathbf{A} \times (\nabla \phi),
\]
con \(\phi = f(r)\) y \(\mathbf{A} = \mathbf{r}\).

Primero, \(\nabla \times \mathbf{r} = \mathbf{0}\) porque:
\[
\nabla \times \mathbf{r} = 
\begin{vmatrix}
\mathbf{i} & \mathbf{j} & \mathbf{k} \\
\frac{\partial}{\partial x} & \frac{\partial}{\partial y} & \frac{\partial}{\partial z} \\
x & y & z
\end{vmatrix}
= \left(\frac{\partial z}{\partial y} - \frac{\partial y}{\partial z}\right)\mathbf{i} - \left(\frac{\partial z}{\partial x} - \frac{\partial x}{\partial z}\right)\mathbf{j} + \left(\frac{\partial y}{\partial x} - \frac{\partial x}{\partial y}\right)\mathbf{k} = \mathbf{0}.
\]

Segundo, \(\nabla \phi = \nabla f(r) = \frac{f'(r)}{r} \mathbf{r}\) por el inciso 1.

Tercero, \(\mathbf{A} \times (\nabla \phi) = \mathbf{r} \times \left(\frac{f'(r)}{r} \mathbf{r}\right) = \frac{f'(r)}{r} (\mathbf{r} \times \mathbf{r}) = \mathbf{0}\).

Sustituimos:
\[
\nabla \times \mathbf{F} = f(r) (\nabla \times \mathbf{r}) - \mathbf{r} \times (\nabla f(r)) = f(r) \cdot \mathbf{0} - \mathbf{0} = \mathbf{0}.
\]




\begin{align*}
    \nabla^2 f(r) &= 2\,\frac{f'(r)}{r} + f''(r) \tag{8}
\end{align*}

Entonces \(f(r)\) es:
\[
\nabla^2 f(r) = \nabla \cdot (\nabla f(r)) = \nabla \cdot \left(\frac{f'(r)}{r} \mathbf{r}\right).
\]

Usamos el resultado del inciso 2, entonces:
\[
\nabla \cdot (g(r) \mathbf{r}) = 3g(r) + r g'(r).
\]

Calculamos \(g'(r)\):
\[
g(r) = \frac{f'(r)}{r} \Rightarrow g'(r) = \frac{d}{dr}\left(\frac{f'(r)}{r}\right) = \frac{f''(r) \cdot r - f'(r) \cdot 1}{r^2} = \frac{r f''(r) - f'(r)}{r^2}.
\]

Entonces:
\[
\nabla^2 f(r) = 3 \cdot \frac{f'(r)}{r} + r \cdot \frac{r f''(r) - f'(r)}{r^2} = \frac{3f'(r)}{r} + \frac{r f''(r) - f'(r)}{r}
\]
\[
= \frac{3f'(r) + r f''(r) - f'(r)}{r} = \frac{2f'(r) + r f''(r)}{r} = 2 \frac{f'(r)}{r} + f''(r).
\]
