Un granjero que pesa $150 \text{ lb}$ sube un saco de grano que pesa $20 \text{ lb}$ a lo largo de una escalera helicoidal alrededor de un silo cilíndrico de radio $25 \text{ ft}$. Conforme el granjero sube, el grano se sale del saco a razón de $1 \text{ lb}$ por cada $10 \text{ ft}$ de ascenso. ¿Cuánto trabajo realiza el granjero si asciende verticalmente $60 \text{ ft}$ en exactamente cuatro vueltas de la hélice? (\textbf{Sugerencia}: describa como un campo vectorial la fuerza ejercida por el granjero para levantar su propio peso más el peso del saco en cada punto de su trayectoria).\\

La componente vertical de la fuerza es:
\[
F_z = 150 + \left(20 - \frac{z}{10}\right) = 170 - \frac{z}{10} \quad \text{lb}.
\]

El trabajo realizado es la integral de \(F_z\) con respecto a \(z\) desde 0 hasta 60 ft:
\[
W = \int_{0}^{60} \left(170 - \frac{z}{10}\right) dz.
\]

\[
W = \left[ 170z - \frac{z^2}{20} \right]_{0}^{60} = \left(170 \cdot 60 - \frac{60^2}{20}\right) - (0) = 10200 - \frac{3600}{20} = 10200 - 180 = 10020 \quad \text{lb-ft}.
\]

El trabajo del granjero es \(10020\) lb-ft.